%%%%%%%%%%%%%%%%%%%%%%%%%%%%%%%%%%%%%%%%%%%%%%%%%%%%%%%%%%%%%%%%%%%%%%
% LaTeX Template: Designer's CV
%
% Source: http://www.howtotex.com
% 
% Feel free to distribute this example, but please keep the referral
% to HowToTeX.com
% 
% Date: March 2012
%
% Modified by Lim Lian Tze to support multiple pages using fix provided at
% http://www.howtotex.com/templates/creating-a-designers-cv-in-latex/
% Date: November 2014
%%%%%%%%%%%%%%%%%%%%%%%%%%%%%%%%%%%%%%%%%%%%%%%%%%%%%%%%%%%%%%%%%%%%%%
% How to use writeLaTeX: 
%
% You edit the source code here on the left, and the preview on the
% right shows you the result within a few seconds.
%
% Bookmark this page and share the URL with your co-authors. They can
% edit at the same time!
%
% You can upload figures, bibliographies, custom classes and
% styles using the files menu.
%
% If you're new to LaTeX, the wikibook is a great place to start:
% http://en.wikibooks.org/wiki/LaTeX
%
%%%%%%%%%%%%%%%%%%%%%%%%%%%%%%%%%%%%%%%%%%%%%%%%%%%%%%%%%%%%%%%%%%%%%%

%%%%%%%%%%%%%%%%%%%%%%%%%%%%%%%%%%%%%
% Document properties and packages
%%%%%%%%%%%%%%%%%%%%%%%%%%%%%%%%%%%%%
\documentclass[a4paper,12pt,final]{memoir}
\usepackage[utf8]{inputenc}           % кодировка
\usepackage[T2A]{fontenc}             % кодировка исходного текста
\usepackage[english,russian]{babel}   % локализация и переносы
% misc
\renewcommand{\familydefault}{pag}	% font
\pagestyle{empty}					% no pagenumbering
\setlength{\parindent}{0pt}			% no paragraph indentation


% required packages (add your own)
\usepackage{flowfram}										% column layout
\usepackage[top=1cm,left=1cm,right=1cm,bottom=1cm]{geometry}% margins
\usepackage{graphicx}										% figures
\usepackage{url}											% URLs
\usepackage[usenames,dvipsnames]{xcolor}					% color
\usepackage{multicol}										% columns env.
	\setlength{\multicolsep}{0pt}
\usepackage{paralist}										% compact lists
\usepackage{tikz}

%%%%%%%%%%%%%%%%%%%%%%%%%%%%%%%%%%%%%
% Create column layout
%%%%%%%%%%%%%%%%%%%%%%%%%%%%%%%%%%%%%
% define length commands
\setlength{\vcolumnsep}{\baselineskip}
\setlength{\columnsep}{\vcolumnsep}

% left frame
\newflowframe{0.2\textwidth}{\textheight}{0pt}{0pt}[left]
	\newlength{\LeftMainSep}
	\setlength{\LeftMainSep}{0.2\textwidth}
	\addtolength{\LeftMainSep}{1\columnsep}
 
% small static frame for the vertical line
\newstaticframe{1.5pt}{\textheight}{\LeftMainSep}{0pt}
 
% content of the static frame
\begin{staticcontents}{1}
\hfill
\tikz{%
	\draw[loosely dotted,color=RoyalBlue,line width=1.5pt,yshift=0]
	(0,0) -- (0,\textheight);}%
\hfill\mbox{}
\end{staticcontents}
 
% right frame
\addtolength{\LeftMainSep}{1.5pt}
\addtolength{\LeftMainSep}{1\columnsep}
\newflowframe{0.7\textwidth}{\textheight}{\LeftMainSep}{0pt}[main01]


%%%%%%%%%%%%%%%%%%%%%%%%%%%%%%%%%%%%%
% define macros (for convience)
%%%%%%%%%%%%%%%%%%%%%%%%%%%%%%%%%%%%%
\newcommand{\Sep}{\vspace{1.5em}}
\newcommand{\SmallSep}{\vspace{0.5em}}

\newenvironment{AboutMe}
	{\ignorespaces\textbf{\color{RoyalBlue} About me}}
	{\Sep\ignorespacesafterend}
	
\newcommand{\CVSection}[1]
	{\Large\textbf{#1}\par
	\SmallSep\normalsize\normalfont}

\newcommand{\CVItem}[1]
	{\textbf{\color{RoyalBlue} #1}}


%%%%%%%%%%%%%%%%%%%%%%%%%%%%%%%%%%%%%
% Begin document
%%%%%%%%%%%%%%%%%%%%%%%%%%%%%%%%%%%%%
\begin{document}

% Left frame
%%%%%%%%%%%%%%%%%%%%
%
% Upload your own photo using the files menu
\begin{figure}
	\hfill
	\includegraphics[width=0.6\columnwidth]{cv-photo.png}
	\vspace{-7cm}
\end{figure}

\begin{flushright}\small
	\url{viclyapunov@ya.ru}  \\
	\url{www.website.com} \\
	+7 (915) 227-96-85
\end{flushright}\normalsize
\framebreak


% Right frame
%%%%%%%%%%%%%%%%%%%%
\Huge\bfseries {\color{RoyalBlue} Виктор Ляпунов} \\
\Large\bfseries  Специалист технической поддержки \\

\normalsize\normalfont


% Experience
\CVSection{Опыт работы}
\CVItem{Август 2017 -  настоящее время, aviata.kz}\\
Техническая поддержка пользователей\\
Регистрация и оформление обращений в системе jira\\
Регистрация пользователей в информационных системах\\
Обеспечение работы IP-телефонии (asterisk)\\
Работа с отделом бухгалтерии по учету офисной техники\\
Закупка расходных материалов и периферийного оборудования\\
Установка и настройка сети Wi-Fi\\
Поддержание доступности интернет-каналов, взаимодействие с операторами связи
\SmallSep

\CVItem{Май 2017 - Июль 2017, Mail.ru Group}\\
Поддержание работоспособности парка вычислительной техники на рабочих местах сотрудников компании\\
Подготовка новых рабочих мест\\
Регистрация пользователей в информационных системах\\
Устранение неисправностей программного и аппаратного обеспечения\\
Учет компьютерной техники и ПО\\
Обеспечение информационной безопасности
\SmallSep

\CVItem{Июнь 2014 - Ноябрь 2016, Яндекс}\\
Техническая поддержка сотрудников компании в части программного и аппаратного обеспечения\\
Установка и поддержка периферийных устройств (принтеры, сканеры)\\
Установка и поддержка сетевых устройств Cisco, серверов (аппаратные и виртуальные серверы, ОС GNU Linux)\\
Решение задач в соответствии с принятыми SLA\\
Обработка тикетов в порядке очередности с учетом приоритетов\\
Учет активов и элементов ИТ-инфраструктуры в системе Oracle Enterprise Business Suite\\
Управление учетными записями в ActiveDirectory\\
Составление ежеквартальных отчетов о проделанной работе
\SmallSep

\CVItem{Сентябрь 2009 - Август 2013, АО Казкоммерц Секьюритиз}\\
Осуществление технической поддерджки пользователей\\
Системное администрирование (Cisco, ActiveDirectory, DNS)\\
Установка, конфигурация, резервное копирование, обновление серверного оборудования и программного обеспечения\\
Осуществление бесперебойной работы серверного оборудования\\
Мониторинг рынка и покупка офисного ИТ-оборудования в соответствии с установленным бюджетом\\
Администрирование корпоративного web-сайта

\Sep

% You'll need these 3 lines at the end of each page!
\clearpage
\framebreak
\framebreak

% Education
\CVSection{Образование}
\CVItem{2013 - 2015, Высшая Школа Экономики, Москва}\\
Магистратура по программе "Информационные системы и сети"
\SmallSep

\CVItem{2011 - 2013, Казахский Национальный технический университет}\\
Магистратура по специальности "Информатика"
\SmallSep

\CVItem{2007 - 2011, Казахский Национальный технический университет}\\
Бакалавриат по специальности "Информатика". Окончил с отличием.
\Sep

\CVSection{Владение языками}
Русский - родной\\
Английский - свободно (Upper Intermediate)\\
Французский - на уровне чтения проф.литературы\\


% Skills
\CVSection{Технические навыки}
\CVItem{Платформы}
\begin{multicols}{3}
\begin{compactitem}[\color{RoyalBlue}$\circ$]
	\item Windows 
	\item Mac OS
	\item Linux
\end{compactitem}
\end{multicols}
\SmallSep

\CVItem{Computer software}
\begin{multicols}{3}
\begin{compactitem}[\color{RoyalBlue}$\circ$]
	\item Lorem 
	\item Ipsum 
	\item Dolor 
	\item Sit 
	\item Amet
	\item Consectetur 
	\item Adipiscing 
	\item Elit
	\item \ldots
\end{compactitem}
\end{multicols}
\Sep 

% References
\CVSection{References}
References upon request.

%%%%%%%%%%%%%%%%%%%%%%%%%%%%%%%%%%%%%
% End document
%%%%%%%%%%%%%%%%%%%%%%%%%%%%%%%%%%%%%
\end{document}